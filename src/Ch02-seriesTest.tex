\begin{center}
	{\Large\bf 单元测验二:无穷级数}
	
	(时间:100分钟)
\end{center}

{\bf 一(每题6分)},判断下列级数是否收敛,并证明
\begin{enumerate}[(1)]
  \setlength{\itemindent}{1cm}
  \item $\sumn\df{1}{\sqrt[3]{n^2+n}}$
  \item $\sumn\df1{n\sqrt[n]n}$
  \item $\sumn\df{\ln n}{n^{3/2}}$
  \item $\sumn\df{n^2}{2^n}$
  \item $\sumn\df{n^2}{\left(2+\df1n\right)^n}$
  \item $\sumn\df1{3^n}\left[\sqrt2+(-1)^n\right]^n$
  \item $\sumn\df1{2^{\ln n}}$
  \item $\sumn e^{-\sqrt[3]n}$
  \item $\sumn\df{n^{n+\frac1n}}{\left(n+\df1n\right)^n}$
\end{enumerate}

{\bf 二(每题8分)}讨论下列级数的敛散性
\begin{enumerate}[(1)]
  \setlength{\itemindent}{1cm}
  \item $\sumn(-1)^n\ln\left(1+\df1n\right)$
  \item $\sumn\df{(-1)^n}{\sqrt n+(-1)^{n+1}}$
  \item $\sumn\sin(\pi\sqrt{n^2+1})$\quad[提示:已知$\limx{0}\df{\sin x}x=1$]
\end{enumerate}

{\bf 三(每题8分)}求下列级数的和
\begin{enumerate}[(1)]
  \setlength{\itemindent}{1cm}
  \item $\sumn\df1{n(n+1)(n+2)}$
  \item $\sumn\df n{3^n}$
\end{enumerate}

{\bf 四(6分)}证明数列$\left\{1+\df1{\sqrt2}+\ldots
+\df1{\sqrt n}-2\sqrt n\right\}$收敛。

\newpage

\begin{center}
	{\Large\bf 参考解答与评分标准}
\end{center}

{\bf 一},解:(1)
$$\limn\df{\df{1}{\sqrt[3]{n^2+n}}}{\df1{n^{2/3}}}=1,$$
由比较判别法,原级数与$\sumn\df1{n^{2/3}}$同敛散,故发散。\hfill(+6)

(2)
$$\limn\df{\df1{n\sqrt[n]n}}{\df1n}=1,$$
由比较判别法,原级数与$\sumn\df1{n}$同敛散,故发散。\hfill(+6)

(3)
注意到$\limn\df{\ln n}{n^{1/4}}=0$,故当$n$充分大时,必有
$$\df{\ln n}{n^{1/4}}<1,\eqno{(+3)}$$
进而
$$\df{\ln n}{n^{3/2}}=\df{\ln n}{n^{1/4}}\df1{n^{5/4}}<\df1{n^{5/4}},$$
$\sumn\df1{n^{5/4}}$收敛,故由比较判别法,原级数收敛。\hfill(+3)

(4)
$$\limn\sqrt[n]{\df{n^2}{2^n}}=\df12<1,$$
由根式判别法,级数收敛。\hfill(+6)

(5)
$$\limn\sqrt[n]{\df{n^2}{\left(2+\df1n\right)^n}}=\df12,$$
由根式判别法,级数收敛。\hfill(+6)

(6)记$a_n=\df1{3^n}\left[\sqrt2+(-1)^n\right]^n$,
$$\df{a_{n+1}}{a_n}=\df{\sqrt2+(-1)^{n+1}}3\leq\df{\sqrt2+1}3<1,$$
由比值判别法(不等式形式),原级数收敛。\hfill(+6)

(7)
$$\df1{2^{\ln n}}=\df1{n^{\ln2}},$$
该级数为$p$-级数,$\ln2<1$,故发散。\hfill(+6)

(8)注意到$\limx{\infty}\df{2x^4}{e^x}=0$,故当$n$充分大时,必有
$$e^{\sqrt[3]n}>2n^{4/3},\eqno{(+3)}$$
从而
$$e^{-\sqrt[3]n}<\df12n^{-4/3},$$
级数$\sumn\df12n^{-4/3}$收敛,故由比较判别法,原级数收敛。\hfill(+3)

(9)
$$\df{n^{n+\frac1n}}{\left(n+\df1n\right)^n}
>\df{n^{n+\frac1n}}{\left(n+1\right)^n}
=\df{\sqrt[n]n}{\left(1+\df1n\right)^n}\to\df1e\;(n\to\infty)$$
由此可知级数通项的极限不为$0$,从而由级数收敛的必要条件可知,该级数发散。\hfill(+6)

{\bf 二},解:(1)
$$\ln(n+2)-\ln(n+1)<\ln(n+1)-\ln n,$$
且
$$\limn\ln\left(1+\df1n\right)=0,$$
故$\left\{\ln\left(1+\df1n\right)\right\}$单调递减趋于零,从而由Leibniz判别法,
原级数收敛。\hfill(+4)
$$\limn\df{\ln\left(1+\df1n\right)}{\df1n}=1,$$
$\sumn\df1n$发散,故由比较判别法,可知$\sumn\ln\left(1+\df1n\right)$发散。
综上,该级数条件收敛。\hfill(+4)

(2)
% $$limn\df{\left|\df{(-1)^n}{\sqrt n+(-1)^{n+1}}\right|}{\df1n}=1,$$
% 由比较判别法$\sumn\left|\df{(-1)^n}{\sqrt n+(-1)^{n+1}}\right|$发散。又
$$\df{(-1)^n}{\sqrt n+(-1)^{n+1}}
=\df{(-1)^n(\sqrt n-(-1)^{n+1})}{n-1}=\df{(-1)^n\sqrt n}{n-1}+\df1{n-1},
\eqno{(+4)}$$
由Leibniz判别法,易证$\sumn\df{(-1)^n\sqrt n}{n-1}$收敛,而$\sumn\df1{n-1}$
发散,故原级数发散。\hfill(+4)

(3)
$$\sin(\pi\sqrt{n^2+1})=\sin[\pi(\sqrt{n^2+1}-n)+n\pi]
=(-1)^n\sin\df{\pi}{\sqrt{n^2+1}+n},\eqno{(+2)}$$
显然,$\left\{\sin\df{\pi}{\sqrt{n^2+1}+n}\right\}$单调递减趋于零,
故由Leibniz判别法,原级数收敛。\hfill(+2)
$$\limn\df{\sin\df{\pi}{{\sqrt{n^2+1}+n}}}{\df1n}
=\limn\df{\sin\df{\pi}{\sqrt{n^2+1}+n}}{\df{\pi}{\sqrt{n^2+1}+n}}
\limn\df {n\pi}{\sqrt{n^2+1}+n}=\df{\pi}2,$$
$\sumn\df1n$发散,故由比较判别法,$\sumn\sin(\pi\sqrt{n^2+1})$发散。
综上,该级数条件收敛。\hfill(+4)

{\bf 四},(1)注意到
$$\df1{n(n+1)(n+2)}=\df12\left(\df1n-2\cdot\df1{n+1}+\df1{n+2}\right),\eqno{(+4)}$$
故该级数的部分和
\begin{eqnarray*}
	S_n&=&\sum\limits_{k=1}^n\df1{k(k+1)(k+2)}\\
	&=&\df12\left[\left(\df11-2\cdot\df12+\df13\right)+
	\left(\df12-2\cdot\df13+\df14\right)+\left(\df13-2\cdot\df14+\df15\right)\right.\\
	&&\left.+\ldots+\left(\df1n-2\cdot\df1{n+1}+\df1{n+2}\right)\right]\\
	&=&\df12\left[1-\df12-\df1{n+1}+\df1{n+2}\right]\to\df14\;(n\to\infty).
\end{eqnarray*}
故所求级数的和为$\df14$.\hfill(+4)

(2)记$S_n=\sum\limits_{k=1}^n\df{k}{3^k}=\df13+\df2{3^2}+\ldots+\df n{3^n}$,则
$$\df13S_n=\df1{3^2}+\df2{3^3}+\ldots+\df{n-1}{3^n}+\df{n}{3^{n+1}},\eqno{(+2)}$$
进而
\begin{align}
	\df23S_n&=\df13+\df1{3^2}+\ldots+\df1{3^n}-\df{n}{3^{n+1}}\notag\\
	&=\df12\left(1-\df1{3^n}\right)-\df{n}{3^{n+1}}\to\df12\;(n\to\infty),\notag
\end{align}
故所求级数的和为$\df34$.\hfill(+6)

{\bf 四},证:记$x_n=1+\df1{\sqrt2}+\ldots+\df1{\sqrt n}-2\sqrt n$,则
$$x_{n+1}-x_n=\df1{\sqrt{n+1}}-2(\sqrt{n+1}-\sqrt n)
=\df{-1}{\sqrt{n+1}(\sqrt{n+1}+\sqrt n)^2}.\eqno{(+3)}$$
注意到
$$\limn\df{\df{1}{\sqrt{n+1}(\sqrt{n+1}+\sqrt n)^2}}{\df1{n^{3/2}}}=\df14,$$
$\sumn\df1{n^{3/2}}$收敛,故由比较判别法,级数$\sumn(x_{n+1}-x_n)$收敛,\hfill(+3)

从而其部分和数列
$$S_n=\sum\limits_{k=1}^n(x_{k+1}-x_k)=x_{n+1}-x_1$$
收敛,由此易知$\{x_n\}$收敛。\hfill(+2)