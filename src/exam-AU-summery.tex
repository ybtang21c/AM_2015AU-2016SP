{\large\bf An UNOFFICIAL Summary for Advanced Mathematics, 

Fall, 2014}
\bigskip

{\large\bf Part 1,试题分析}
\bigskip

{\bf 一、填空(15分,每题3分)}:平均分10.11

\begin{enumerate}
  \setlength{\itemindent}{1cm}
  \item 用定积分的定义计算极限,共出现3种不同的错误答案
  \item 无穷小量的比较(课后作业),共出现4种不同的错误答案
  \item 考察导数的定义,以及对形如$\limx{a}\df{f(x)}{x-a}$的理解,
  前两年考过类型完全一样的题目,共出现22种不同的错误答案,有2人空白,
  是错误率最高的一题,有些意外!
  \item 考察利用对称性计算定积分,课堂上讲过类似的题目,相对于其他学员队
  (很多人空白),我们的得分率较高,但也出现了22种不同的错误答案,有2人空白,
  错误率在填空题中排名第二{\it(居然不是第一!?)}
  \item 考察旋转体的体积公式(以及无穷积分的计算),属于基本题,出现了
  10种不同的错误答案
\end{enumerate}

{\bf 二、选择(15分,每题3分)}:平均分12.43

\begin{enumerate}
  \setlength{\itemindent}{1cm}
  \item 基本的极限计算题,课后作业有类似的题目,3人选A,13人选D
  \item 考察函数极限以及连续的定义,5人选B,6人选D
  \item 渐近线的计算,错误率最低的一题,3人选A
  \item 错误率最高的题目之一,可以使用分部积分或者定积分的几何意义计算
  {\it (课堂上讲过有关定积分几何意义的专题,没想到真考了)},5人选A、
  5人选C、20人选D,居然有1人没选{\it (不知道脑子怎么想的,蒙一下
  总是可以的啊!!!!)}
  \item 错误率和前一题相当,考察变限积分和极值点的判定,有一定的难度,
  3人选B(可能只算了$x=0$),29人选C(应该是漏算了$x=0$)
  {\it (关于极值点和拐点的判定,特别是多项式函数有关的问题,今后有必要
  考虑做个专题讲一下)}
\end{enumerate}

{\bf 三、(6分)}:平均分5.02

考察变限积分和参数方程求导,很多人都是二阶导数算错(主要是漏算了$\df{\d t}
{\d x}$),还有少数人$\df{\d y}{\d t}$算错。

{\bf 四、(6分)}:平均分4.77

考研难度的极限计算问题,化成$e^{f(x)}$的形式计算,需要用到无穷小代换和
L'Hospital法则。主要的错误,将底数的分子$\ln(1+x)$直接代换成了$x$!有1人完全未动笔!
{\it (极限计算是第一学期的基础题,必须熟练掌握,以后需要进一步加强练习,
可以考虑在学期末增加一个极限综合测验)}

{\bf 五、(6分)}:平均分5.77

考察隐函数求导和曲率计算公式,没什么难度。得分率如此高,也许是因为在课上曾提到
渐近线和曲率通常都会考。1人空白!

{\bf 六、(6分)}:平均分3.95

第三次考到了变限积分求导的问题,由于被积式中包含了$x$,需要先通过变换$t=x-u$
将其化成可以套用公式的形式,然后连续求导即可。7人完全没想到要用变换,4人根本没动笔!
{\it (变限积分的求导邻近学期末才讲,为了保证大家熟练掌握,有必要多强调几次!另外,
第二学期学习重积分,可能也要用到这部分的知识,到时再强调一次!)}

{\bf 七、(6分)}:平均分3.95

基础题的最后一题,至此卷面共60分。关键在于构造函数$F(x)=(2-x)f(x)$,然后使用
Rolle定理,其中为了构成Rolle定理的条件还需要用到定积分中值定理。可喜的是,大部分人都正确理解了
$\dint_0^1f(x)\d x$这个条件{\it (如果没记错,这部分的内容是请胡老师代课讲
的:P)},但仍有7人整题未动笔。

{\bf 八、(8分)}:平均分7.41

送分题。只要能看懂条件,基本都能算对。2人没看懂条件,1人求导算错,1人未动笔。

{\bf 九、(8分)}:平均分6.37

坑爹排名第二的题目,但得分率仍然不低,多数人丢分的原因是未讨论$\lambda<0$的情况。

{\bf 十、(8分)}:平均分4.40

稳稳打进近10年坑爹排行前三甲,保守估计排名第二!涌现出了各种奇葩解答,包括画出了
整个登月过程的“大神”!主要的问题是看不懂题目{\it (题目本身的叙述也很含混)},
全校得满分的估计不超过10人{\it (居然真的有!)},本班无一人全对,最高分6分。
此外,考虑到是应用题,随便设几个符号,写几句话都有点分,所以得零分的也很少。
{\it (P.S. 能有耐心看完本题解答并且看明白的,也算是神人了,不信就试试!)}

{\bf 十一、(8分)}:平均分3.85

挺有意思的一题,不过所有的难点只在于对$|\ln x|^n$的处理,15人未动笔或写的文不
对题,原因主要在此。除此之外,本题须用到分部积分{\it (利用分部积分进行递推
今后有必要作为专题强调一下)},第一问解决了,第二问求极限等于是白送的。

{\bf 十二、(8分)}:平均分3.36

此题全班无人满分,且得分率最低!第二问30人完全没有动手。本题解法非常多样,是一道
综合训练导数应用的好题目。

\bigskip
{\large\bf Part 2,成绩概况}

\begin{enumerate}[(1)]
  \setlength{\itemindent}{1cm}
  \item 111人全部参加考试,卷面平均分71.40,及格率84.68\%
  \item 最终成绩按照卷面80\%,平时20\%综合计算,平均分74.27,及格率97.30\%
  \item 综合成绩分布如下,90分以上12人,80-89分34人,70-79分30人,70分以下33人
  \item 综合成绩最高分96分
\end{enumerate}

\bigskip
{\large\bf Part 3,分析小结}

\begin{enumerate}[(1)]
  \setlength{\itemindent}{1cm}
  \item 本次考试难度略高{\it (请注意,只是略高,而不是很高!)},据本人掌握的情况,
  至少有4个班次(含高班和钱班)的平均分超过80,成绩最好的平均分达到83以上,
  我们的差距不小
  \item 在技术类普班中,我们的成绩处于中等水平,略偏下,但较之六院历年的成绩
  表现,仍有明显进步,特别是班内的成绩分布较过去更平均
  \item 从卷面情况看,许多同学对考试中的解题表现出了一些的不适应,例如不能
  正确理解题意、对条件的处理不熟练、步骤不够规范等,原因可能还是平时测验进行的较少,
  今后需要有意识地在这方面有所加强{\it (1-2章进行一次单元测验)}
  \item 从考试成绩和平时成绩的匹配程度来看,大部分平时作业较好的同学考试也发挥
  得较好{\it (说明:平时成绩以作业打分为基础,取5个轮次中每个轮次的最高分,然后
  去掉得分最低的一个轮次,再加以转换综合;最终全班平时成绩的平均分为85.66,且
  每人所得平均成绩均不低于此次期末考试的卷面成绩)},但也有一些偏离情况严重的个人,
  今后须加以重点关注
  \item 特别表扬两位同学:汪天翔、王艺霖,都因为各种原因耽误了相当长时间的学业,
  但都能自己埋头苦干,勤奋用功,一边学习新的知识一边补习前面落下的内容,补全并且
  自己订正了所有的课后作业,期末考试的成绩证明了他们的努力,值得大家学习!
  \item 通过微信群答疑看来也起到了一定的正面效果,但新学期开学后以及今后的班级
  中还能否保持这种可以随时交流的状态,仍存在诸多不定因素,如何更好地开展师生交流互动,
  需要请大家继续集思广益!
\end{enumerate}


